\section{Software Register Details}
\label{sec:swreg_details}

This section presents each software register in detail.

\begin{table}[H]
  \centering
  \begin{tabularx}{\textwidth}{|l|l|X|}
    
    \hline
    \rowcolor{iob-green}
    {\bf Bit \#} & {\bf R/W} & {\bf Description} \\ \hline

    31-17   &   & Reserved \\ \hline
    \rowcolor{iob-blue}
    16      & RW & RECSMALL - Receive Small Packets. 

                0 = Packets smaller than MINFL are ignored; 

                1 = Packets smaller than MINFL are accepted. \\ \hline
    15      & RW &   PAD - Padding enabled. 

                0 = do not add pads to short frames; 

                1 = add pads to short frames (until the minimum frame length is equal to MINFL). \\ \hline
    \rowcolor{iob-blue}
    14      & RW &   HUGEN = Huge Packets Enable. 

                0 = the maximum frame length is MAXFL. All Additional bytes are discarded; 

                1 = Frames up 64KB are transmitted. \\ \hline
    13      & RW &   CRCEN - CRC Enable. 

                0 = Tx MAC does not append the CRC (passed frames already contain the CRC; 

                1 = Tx MAC appends the CRC to every frame. \\ \hline
    \rowcolor{iob-blue}
    12      & RW &   DLYCRCEN - Delayed CRC Enabled. 

                0 = Normal operation (CRC calculation starts immediately after the SFD); 

                1 = CRC calculation starts 4 bytes after the SFD. \\ \hline
    11      & RW &  Reserved \\ \hline
    \rowcolor{iob-blue}
    10      & RW &  FULLD - Full Duplex. 

                0 = Half duplex mode; 

                1 = Full duplex mode. \\ \hline
    9       & RW &  EXDFREN - Excess Defer Enabled. 

                0 = When the excessive deferral limit is reached, a packet is aborted; 

                1 = MAC waits for the carrier indefinitely. \\ \hline
    \rowcolor{iob-blue}
    8       & RW &  NOBCKOF - No Backoff. 

                0 = Normal operation (a binary exponential backoff algorithm is used); 

                1 = Tx MAC starts retransmitting immediately after the collision. \\ \hline
    7       & RW &  LOOPBCK - Loop Back. 

                0 = Normal operation; 

                1 = Tx is looped back to the RX. \\ \hline
    \rowcolor{iob-blue}
    6       & RW &  IFG - Interframe Gap for Incoming frames. 

                0 = Normal operation (minimum IFG is required for a frame to be accepted; 

                1 = All frames are accepted regardless to the IFG. \\ \hline
    5       & RW &  PRO - Promiscuous. 

                0 = Check the destination address of the incoming frames; 

                1 = Receive the frame regardless of its address. \\ \hline
    \rowcolor{iob-blue}
    4       & RW &   IAM - Individual Address Mode. 

                0 = Normal operation (physical address is checked when the frame is received); 

                1 = The individual hash table is used to check all individual addresses received. \\ \hline
    3       & RW &   BRO - Broadcast Address. 

                0 = Receive all frames containing the breadcast address; 

                1 = Reject all frames containing the broadcast address unless the PRO bit=1. \\ \hline
    \rowcolor{iob-blue}
    2       & RW &   NOPRE - No Preamble. 

                0 = Normal operation (7-byte preamble); 

                1 = No preamble is sent. \\ \hline
    1       & RW &   TXEN - Transmit Enable. 

                0 = Transmit is disabled; 

                1 = Transmit is enabled. 

                If the value, written to the TX\_BD\_NUM register, is equal to 0x0 (zero buffer descriptors are used), then the transmitter is automatically disabled regardless of the TXEN bit. \\ \hline
    \rowcolor{iob-blue}
    0       & RW &   RXEN - Receive Enable. 

                0 = Transmit is disabled; 

                1 = Transmit is enabled. 

                If the value, written to the TX\_BD\_NUM register, is equal to 0x80 (all buffer descriptors are used for transmit buffer descriptors, so there is no receive BD), then the receiver is automatically disabled regardless of the RXEN bit.
 \\ \hline
 
  \end{tabularx}
    \caption{MODER (Mode Register)}
  \label{swreg_details:moder}
\end{table}

\begin{table}[H]
  \centering
  \begin{tabularx}{\textwidth}{|l|l|X|}
    
    \hline
    \rowcolor{iob-green}
    {\bf Bit \#} & {\bf R/W} & {\bf Description} \\ \hline

    31-7   &   & Reserved \\ \hline
    \rowcolor{iob-blue}
    6      & RW & RXC - Receive Control Frame

                This bit indicates that the control frame was received. It is
                cleared by writting 1 to it. Bit RXFLOW (CTRLMODER register)
                must be set to 1 in order to get the RXC bit set. \\ \hline
    5      & RW & TXC - Transmit Control Frame

                This bit indicates that a control frame was transmitted. It is
                cleared by writting 1 to it. Bit RXFLOW (CTRLMODER register)
                must be set to 1 in order to get the TXC bit set. \\ \hline
    \rowcolor{iob-blue}
    4      & RW & BUSY - Busy

                This bit indicates that a buffer was received and discarded due
                to a lack of buffers. It is cleared by writing 1 to it. This
                bit appears regardless to the IRQ bits in the Receive or
                Transmit Buffer Descriptors.\\ \hline
    3      & RW & RXE - Receive Error

                This bit indicates that an error occurred while receiving data.
                It is cleared by writing 1 to it. This bit appears only when
                IRQ bit is set in the Receive Buffer Descriptor.\\ \hline
    \rowcolor{iob-blue}
    2      & RW & RXB - Receive Frame

                This bit indicates that a frame was received. It is cleared by
                writing 1 to it. This bit appears only when IRQ bit is set in
                the Receive Buffer Descriptor. If a control frame is received,
                then RXC bit is set instead of the RXB bit. (CTRLMODER (Control
                Module Mode Register) description for more details.)\\
                \hline
    1      & RW & TXE - Transmit Error

                This bit indicates that a buffer was not transmitted due to a
                transmit error. It is cleared by writing 1 to it. This bit
                appears only when IRQ bit is set in the Receive Buffer
                Descriptor. This bit appears only when IRQ bit is set in the
                Transmit Buffer Descriptor.\\ \hline
    \rowcolor{iob-blue}
    0      & RW & TXB - Transmit Buffer

                This bit indicates that a buffer has been transmitted. It is
                cleared by writing 1 to it. This bit appears only when IRQ bit
                is set in the Transmit Buffer Descriptor.\\ \hline
  \end{tabularx}
    \caption{INT\_SOURCE (Interrupt Source Register)}
  \label{swreg_details:int_source}
\end{table}

\begin{table}[H]
  \centering
  \begin{tabularx}{\textwidth}{|l|l|X|}
    
    \hline
    \rowcolor{iob-green}
    {\bf Bit \#} & {\bf R/W} & {\bf Description} \\ \hline

    31-7   &   & Reserved \\ \hline
    \rowcolor{iob-blue}
    6      & RW & RXC\_M - Receive Control Frame Mask
                
                0 = Event masked

                1 = Event causes an interrupt \\ \hline
    5      & RW & TXC - Transmit Control Frame Mask

                0 = Event masked

                1 = Event causes an interrupt \\ \hline
    \rowcolor{iob-blue}
    4      & RW & BUSY - Busy Mask

                0 = Event masked

                1 = Event causes an interrupt \\ \hline
    3      & RW & RXE - Receive Error Mask

                0 = Event masked

                1 = Event causes an interrupt \\ \hline
    \rowcolor{iob-blue}
    2      & RW & RXB - Receive Frame Mask

                0 = Event masked

                1 = Event causes an interrupt \\ \hline
   1      & RW & TXE - Transmit Error Mask

                0 = Event masked

                1 = Event causes an interrupt \\ \hline
    \rowcolor{iob-blue}
    0      & RW & TXB - Transmit Buffer Mask

                0 = Event masked

                1 = Event causes an interrupt \\ \hline
  \end{tabularx}
    \caption{INT\_MASK (Interrupt Mask Register)}
  \label{swreg_details:int_mask}
\end{table}

\begin{table}[H]
  \centering
  \begin{tabularx}{\textwidth}{|l|l|X|}
    
    \hline
    \rowcolor{iob-green}
    {\bf Bit \#} & {\bf R/W} & {\bf Description} \\ \hline

    31-7   &   & Reserved \\ \hline
    \rowcolor{iob-blue}
    6-0      & RW & IPGT - Back to Back Inter Packet Gap

                Full Duplex: The recommended value is 0x15, which equals
                $0.96\mu s$ IPG (100 Mbps) or $9.6\mu s$ (10 Mbps). The desired
                period in nibble times minus 6 should be written to the
                register.

                Half Duplex: The recommended value is 0x12, which equals
                $0.96\mu s$ IPG (100 Mbps) or $9.6\mu s$ (10 Mbps). The desired
                period in nibble times minus 3 should be written to the
                register. \\ \hline
  \end{tabularx}
    \caption{IPGT (Back to Back Inter Packet Gap Register)}
  \label{swreg_details:ipgt}
\end{table}

\begin{table}[H]
  \centering
  \begin{tabularx}{\textwidth}{|l|l|X|}
    
    \hline
    \rowcolor{iob-green}
    {\bf Bit \#} & {\bf R/W} & {\bf Description} \\ \hline

    31-7   &   & Reserved \\ \hline
    \rowcolor{iob-blue}
    6-0      & RW & IPGR1 - Non Back to Back Inter Packet Gap 1

                When a carrier sense appears within the IPGR1 window, Tx MAC
                defers and the IPGR counter is reset. When a carrier sense
                appears later than the IPGR1 window, the IPGR counter continues
                counting. The recommended and default value for this register
                is 0xC. It must be within the range [0,IPGR2]. \\ \hline
  \end{tabularx}
    \caption{IPGR1 (Non Back to Back Inter Packet Gap Register 1)}
  \label{swreg_details:ipgr1}
\end{table}

\begin{table}[H]
  \centering
  \begin{tabularx}{\textwidth}{|l|l|X|}
    
    \hline
    \rowcolor{iob-green}
    {\bf Bit \#} & {\bf R/W} & {\bf Description} \\ \hline

    31-7   &   & Reserved \\ \hline
    \rowcolor{iob-blue}
    6-0      & RW & IPGR2 - Non Back to Back Inter Packet Gap 2

                The recommended and default value is 0x12, which equals to $0.96
                \mu s$ IPG (100 Mbps) or $9.6 \mu s$ (10 Mbps).\\ \hline
  \end{tabularx}
    \caption{IPGR2 (Non Back to Back Inter Packet Gap Register 2)}
  \label{swreg_details:ipgr2}
\end{table}

\begin{table}[H]
  \centering
  \begin{tabularx}{\textwidth}{|l|l|X|}
    
    \hline
    \rowcolor{iob-green}
    {\bf Bit \#} & {\bf R/W} & {\bf Description} \\ \hline

    31-16   & RW  & MINFL - Minimum Frame Length 

                The minimum Ethernet packet is 64 bytes long. If a reception of
                smaller frames is needed, assert the RECSMALL bit (in the mode
                register MODER) or change the value of this register. To
                transmit small packets, assert the PAD bit or the MINFL value
                (see the PAD bit description in the MODER register). \\ \hline
    \rowcolor{iob-blue}
    15-0   & RW  & MAXFL - Maximum Frame Length 

                The maximum Ethernet packet is 1518 bytes long. To support this
                and to leave some additional space for the tags, a default
                maximum packet length equals 1536 bytes (0x0600). If there is a
                need to support bigger packets, you can assert the HUGEN bit or
                increase the value of the MAXFL field (see the HUGEN bit
                description in the MODER).\\ \hline
  \end{tabularx}
    \caption{PACKETLEN (Packet Length Register)}
  \label{swreg_details:packetlen}
\end{table}

\begin{table}[H]
  \centering
  \begin{tabularx}{\textwidth}{|l|l|X|}
    
    \hline
    \rowcolor{iob-green}
    {\bf Bit \#} & {\bf R/W} & {\bf Description} \\ \hline

    31-20   &   & Reserved \\ \hline
    \rowcolor{iob-blue}
    19-16      & RW & MAXRET - Maximum Retry

                This field specifies the maximum number of consequential
                retransmission attempts after the collision is detected. When
                the maximum number has been reached, the Tx MAC reports an
                error and stops transmitting the current packet. According to
                the Ethernet standard, the MAXRET default value is set to 0xf
                (15).\\ \hline
    15-6      & RW &   Reserved \\ \hline
    \rowcolor{iob-blue}
    5-0       & RW &   COLLVALID - Collision Valid 
%
                This field specifies a collision time window. A collision that
                occurs later than the time window is reported as a "Late
                Collisions" and transmission of the current packet is aborted.
                The default value equals 0x3f (by default, a late collision is
                every collision that occurs 64 bytes ($63 + 1$) from the
                preamble) \\ \hline
  \end{tabularx}
    \caption{COLLCONF (Collision and Retry Configuration Register)}
  \label{swreg_details:collconf}
\end{table}

\begin{table}[H]
  \centering
  \begin{tabularx}{\textwidth}{|l|l|X|}
    
    \hline
    \rowcolor{iob-green}
    {\bf Bit \#} & {\bf R/W} & {\bf Description} \\ \hline

    31-8   & RW  & Reserved \\ \hline
    \rowcolor{iob-blue}
      7-0   & RW  & Transmit Buffer Descriptor (Tx BD) Number

                Number of the Tx BD. Number of the Rx BD equals to the (0x80 –
                Tx BD number). Maximum number of the Tx BD is 0x80. Values
                greater then 0x80 cannot be written to this register
                (ignored).\\ \hline
  \end{tabularx}
    \caption{TX\_BD\_NUM (Transmit BD Number Register)}
  \label{swreg_details:tx_bd_num}
\end{table}

\begin{table}[H]
  \centering
  \begin{tabularx}{\textwidth}{|l|l|X|}
    
    \hline
    \rowcolor{iob-green}
    {\bf Bit \#} & {\bf R/W} & {\bf Description} \\ \hline

    31-3   & RW  & Reserved \\ \hline
    \rowcolor{iob-blue}
    2  & RW  & TXFLOW - Transmit Flow Control

                0 = PAUSE control frames are blocked. 

                1 = PAUSE control frames are allowed to be sent. This bit
                enables the TXC bit in the INT\_SOURCE register.\\ \hline
    1  & RW  & RXFLOW - Receive Flow Control

                0 = Received PAUSE control frames are ignored.

                1 = The transmit function (Tx MAC) is blocked when a PAUSE
                control frame is received. This bit enables the RXC bit in the
                INT\_SOURCE register.\\ \hline
    \rowcolor{iob-blue}
    0  & RW  & PASSALL - Pass All Receive Frames

                0 = Control frames are not passed to the host. RXFLOW must be
                set to 1 in order to use PAUSE control frames.

                1 = All received frames are passed to the host.\\ \hline
  \end{tabularx}
    \caption{CTRLMODER (Control Module Mode Register)}
  \label{swreg_details:ctrlmoder}
\end{table}

\begin{table}[H]
  \centering
  \begin{tabularx}{\textwidth}{|l|l|X|}
    
    \hline
    \rowcolor{iob-green}
    {\bf PASSALL \#} & {\bf RXFLOW} & {\bf Description} \\ \hline

    0   & 0 &   When a PAUSE control frame is received, nothing happens. The
                control frame is not stored to the memory.\\ \hline
    \rowcolor{iob-blue}
    0   & 1 &   When a PAUSE control frame is received, RXC interrupt is set
                and pause timer is updated. The control frame is not stored to
                the memory.\\ \hline
    1   & 0 &   When a PAUSE control frame is received, it is stored to the
                memory as a normal data frame. RXB interrupt is set (if the
                related buffer descriptor has an IRQ bit set to 1). RXC
                interrupt is not set and pause timer is not updated.\\ \hline
    \rowcolor{iob-blue}
    1   & 1 &   When a PAUSE control frame is received, RXC interrupt is set
                and pause timer is updated. Besides that the control frame is
                also stored to the memory as a normal data frame.\\ \hline
  \end{tabularx}
    \caption{PASSALL and RXFLOW Operation}
  \label{swreg_details:passall_rxflow}
\end{table}

\begin{table}[H]
  \centering
  \begin{tabularx}{\textwidth}{|l|l|X|}
    
    \hline
    \rowcolor{iob-green}
    {\bf Bit \#} & {\bf R/W} & {\bf Description} \\ \hline

    31-9   &   & Reserved \\ \hline
    \rowcolor{iob-blue}
    8      & RW & MIINOPRE - No Preamble

                0 = 32-bit premable sent

                1 = No preamble sent\\ \hline
    7-0      & RW &   CLKDIV - Clock Divider

                The field is a host clock divider factor. The host clock can be
                divided by an even number, greater then 1. The default value is
                0x64 (100).\\ \hline
  \end{tabularx}
    \caption{MIIMODER (MII Mode Register)}
  \label{swreg_details:miimoder}
\end{table}

\begin{table}[H]
  \centering
  \begin{tabularx}{\textwidth}{|l|l|X|}
    
    \hline
    \rowcolor{iob-green}
    {\bf Bit \#} & {\bf R/W} & {\bf Description} \\ \hline

    31-3   &   & Reserved \\ \hline
    \rowcolor{iob-blue}
    2      & RW & WCTRLDATA - Write Control Data \\ \hline
    1      & RW & RSTAT - Read Status \\ \hline
    \rowcolor{iob-blue}
    0      & RW & SCANSTAT - Scan Status \\ \hline
  \end{tabularx}
    \caption{MIICOMMAND (MII Command Register)}
  \label{swreg_details:miicommand}
\end{table}

\textbf{Note: While one operation is in progress, BUSY signal (MIISTATUS
register is set. Next operation can be started after the previous one is
finished (and BUSY signal cleared to zero).}

\begin{table}[H]
  \centering
  \begin{tabularx}{\textwidth}{|l|l|X|}
    
    \hline
    \rowcolor{iob-green}
    {\bf Bit \#} & {\bf R/W} & {\bf Description} \\ \hline

    31-13   &   & Reserved \\ \hline
    \rowcolor{iob-blue}
    12-8    & RW & RGAD - Register Address (within the PHY selected by the
                        FIAD[4:0]) \\ \hline
    7-5     & RW & Reserved \\ \hline
    \rowcolor{iob-blue}
    4-0     & RW & FIAD - PHY Address \\ \hline
  \end{tabularx}
    \caption{MIIADDRESS (MII Address Register)}
  \label{swreg_details:miiaddress}
\end{table}

\begin{table}[H]
  \centering
  \begin{tabularx}{\textwidth}{|l|l|X|}
    
    \hline
    \rowcolor{iob-green}
    {\bf Bit \#} & {\bf R/W} & {\bf Description} \\ \hline

    31-16   &   & Reserved \\ \hline
    \rowcolor{iob-blue}
      15-0    & RW &  CTRLDATA - Control Data (data to be written to the PHY)
                    \\ \hline
  \end{tabularx}
    \caption{MIITX\_DATA (MII Transmit Data)}
  \label{swreg_details:miitx_data}
\end{table}

\begin{table}[H]
  \centering
  \begin{tabularx}{\textwidth}{|l|l|X|}
    
    \hline
    \rowcolor{iob-green}
    {\bf Bit \#} & {\bf R/W} & {\bf Description} \\ \hline

    31-16   &   & Reserved \\ \hline
    \rowcolor{iob-blue}
    15-0    & R &  PRSD - Received Data (data to be read from PHY) \\ \hline
  \end{tabularx}
    \caption{MIIRX\_DATA (MII Receive Data)}
  \label{swreg_details:miirx_data}
\end{table}

\begin{table}[H]
  \centering
  \begin{tabularx}{\textwidth}{|l|l|X|}
    
    \hline
    \rowcolor{iob-green}
    {\bf Bit \#} & {\bf R/W} & {\bf Description} \\ \hline

    31-3   &   & Reserved \\ \hline
    \rowcolor{iob-blue}
    2      & R & NVALID - Invalid

                0 = The data in the MSTATUS register is valid.

                1 = The data in the MSTATUS register is invalid. 

                This bit is only valid when the scan status operation is
                active.\\ \hline
    1      & R & BUSY

                0 = The MII is ready.

                1 = The MII is bysy (operation in progress). \\ \hline
    \rowcolor{iob-blue}
    0      & R & LINKFAIL

                0 = The link is OK.

                1 = The link failed. 

                The link fail condition occurred (now the link might be OK).
                Another status read gets a new status.\\ \hline
  \end{tabularx}
    \caption{MIISTATUS (MII Status Register)}
  \label{swreg_details:miistatus}
\end{table}

\begin{table}[H]
  \centering
  \begin{tabularx}{\textwidth}{|l|l|X|}
    
    \hline
    \rowcolor{iob-green}
    {\bf Bit \#} & {\bf R/W} & {\bf Description} \\ \hline

    31-24   & RW  & Byte 2 of the Ethernet MAC address (individual address) \\
                    \hline
    \rowcolor{iob-blue}
    23-16   & RW  & Byte 3 of the Ethernet MAC address (individual address) \\
                    \hline
    15-8    & RW  & Byte 4 of the Ethernet MAC address (individual address) \\
                    \hline
    \rowcolor{iob-blue}
    7-0     & RW  & Byte 5 of the Ethernet MAC address (individual address) \\
                    \hline
  \end{tabularx}
    \caption{MAC\_ADDR0 (MAC Address Register 0)}
  \label{swreg_details:mac_addr0}
\end{table}

\begin{table}[H]
  \centering
  \begin{tabularx}{\textwidth}{|l|l|X|}
    
    \hline
    \rowcolor{iob-green}
    {\bf Bit \#} & {\bf R/W} & {\bf Description} \\ \hline

    31-16   & RW  & Reserved \\ \hline
    \rowcolor{iob-blue}
    15-8    & RW  & Byte 0 of the Ethernet MAC address (individual address) \\
                    \hline
    7-0     & RW  & Byte 1 of the Ethernet MAC address (individual address) \\
                    \hline
  \end{tabularx}
    \caption{MAC\_ADDR1 (MAC Address Register 1)}
  \label{swreg_details:mac_addr1}
\end{table}

Note: When an address is transmitted, byte 0 is sent first and byte 5 last.

\begin{table}[H]
  \centering
  \begin{tabularx}{\textwidth}{|l|l|X|}
    
    \hline
    \rowcolor{iob-green}
    {\bf Bit \#} & {\bf R/W} & {\bf Description} \\ \hline

    31-0   & RW  & Hash0 value \\ \hline
  \end{tabularx}
    \caption{HASH0 (HASH Register 0)}
  \label{swreg_details:hash0}
\end{table}

\begin{table}[H]
  \centering
  \begin{tabularx}{\textwidth}{|l|l|X|}
    
    \hline
    \rowcolor{iob-green}
    {\bf Bit \#} & {\bf R/W} & {\bf Description} \\ \hline

    31-0   & RW  & Hash1 value \\ \hline
  \end{tabularx}
    \caption{HASH1 (HASH Register 1)}
  \label{swreg_details:hash1}
\end{table}

\begin{table}[H]
  \centering
  \begin{tabularx}{\textwidth}{|l|l|X|}
    
    \hline
    \rowcolor{iob-green}
    {\bf Bit \#} & {\bf R/W} & {\bf Description} \\ \hline

    31-17   &   & Reserved \\ \hline
    \rowcolor{iob-blue}
    16      & RW & TXPAUSERQ - Tx Pause Request

                Writing 1 to this bit starts sending control frame procedure.
                Bit is automatically cleared to zero. \\ \hline
    15-0    & RW & TXPAUSETV - Tx Pause Timer Value

                 The value that is send in the pause control frame.\\ \hline
  \end{tabularx}
    \caption{TXCTRL (Tx Control Register)}
  \label{swreg_details:txctrl}
\end{table}
